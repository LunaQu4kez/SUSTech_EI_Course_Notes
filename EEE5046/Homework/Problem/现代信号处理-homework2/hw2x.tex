\documentclass[UTF8]{ctexart}

\usepackage{fancyhdr}
\usepackage{extramarks}
\usepackage{amsmath}
\usepackage{amsthm}
\usepackage{amsfonts}
\usepackage{tikz}
\usepackage[plain]{algorithm}
\usepackage{algpseudocode}
\usepackage{enumerate}

\usetikzlibrary{automata,positioning}

%
% Basic Document Settings
%

\topmargin=-0.45in
\evensidemargin=0in
\oddsidemargin=0in
\textwidth=6.5in
\textheight=9.0in
\headsep=0.25in

\linespread{1.1}

\pagestyle{fancy}
\lhead{\hmwkAuthorName\ \hmwkStudentID}
\chead{\hmwkClass: \hmwkTitle}
\rhead{\firstxmark}
\lfoot{\lastxmark}
\cfoot{\thepage}

\renewcommand\headrulewidth{0.4pt}
\renewcommand\footrulewidth{0.4pt}

\setlength\parindent{0pt}

%
% Create Problem Sections
%

\newcommand{\enterProblemHeader}[1]{
    \nobreak\extramarks{}{Problem \arabic{#1} continued on next page\ldots}\nobreak{}
    \nobreak\extramarks{Problem \arabic{#1} (continued)}{Problem \arabic{#1} continued on next page\ldots}\nobreak{}
}

\newcommand{\exitProblemHeader}[1]{
    \nobreak\extramarks{Problem \arabic{#1} (continued)}{Problem \arabic{#1} continued on next page\ldots}\nobreak{}
    \stepcounter{#1}
    \nobreak\extramarks{Problem \arabic{#1}}{}\nobreak{}
}

\setcounter{secnumdepth}{0}
\newcounter{partCounter}
\newcounter{homeworkProblemCounter}
\setcounter{homeworkProblemCounter}{1}
\nobreak\extramarks{Problem \arabic{homeworkProblemCounter}}{}\nobreak{}

\newenvironment{homeworkProblem}{
    \section{Problem \arabic{homeworkProblemCounter}}
    \setcounter{partCounter}{1}
    \enterProblemHeader{homeworkProblemCounter}
}{
    \exitProblemHeader{homeworkProblemCounter}
}

%
% Homework Details
%   - Title
%   - Due date
%   - Class
%   - Section/Time
%   - Instructor
%   - Author
%

\newcommand{\hmwkTitle}{Homework 2}
\newcommand{\hmwkDueDate}{Nov. 19, 2025}
\newcommand{\hmwkClass}{现代信号处理}
\newcommand{\hmwkStudentID}{这里写学号}
\newcommand{\hmwkAuthorName}{这里写姓名}

%
% Title Page
%

\title{
    \vspace{2in}
    \textmd{\textbf{\hmwkClass:\ \hmwkTitle}}\\
    \normalsize\vspace{0.1in}\small{Due\ on\ \hmwkDueDate\ }\\
    \vspace{3in}
}

\author{
	\textbf{\hmwkAuthorName}\\
	\textbf{学号\ \hmwkStudentID}
}
\date{}

\renewcommand{\part}[1]{\textbf{\large Part \Alph{partCounter}}\stepcounter{partCounter}}

%
% Various Helper Commands
%

% Useful for algorithms
\newcommand{\alg}[1]{\textsc{\bfseries \footnotesize #1}}

% For derivatives
\newcommand{\deriv}[1]{\frac{\mathrm{d}}{\mathrm{d}x} (#1)}

% For partial derivatives
\newcommand{\pderiv}[2]{\frac{\partial}{\partial #1} (#2)}

% Integral dx
\newcommand{\dx}{\mathrm{d}x}

% Alias for the Solution section header
\newcommand{\solution}{\textbf{\large Solution}}

% Probability commands: Expectation, Variance, Covariance, Bias
\newcommand{\E}{\mathrm{E}}
\newcommand{\Var}{\mathrm{Var}}
\newcommand{\Cov}{\mathrm{Cov}}
\newcommand{\Bias}{\mathrm{Bias}}

\begin{document}

\maketitle

\pagebreak

要求:latex\\
DDL:2025/11/19 下午23:59分前提交pdf电子版\\
电子版以"homework2-姓名-学号"形式发送到12332207@mail.sustech.edu.cn邮箱\\


\begin{homeworkProblem}
    设 \( x[n] = \delta[n] + 2\delta[n - 1] - \delta[n - 3] \) 和 \( h[n] = 2\delta[n + 1] + 2\delta[n - 1] \),计算下列各卷积。
    \begin{enumerate}[(1)]
        \item \( y_1[n] = x[n] * h[n] \)
        \item \( y_2[n] = x[n + 2] * h[n] \)
        \item \( y_3[n] = x[n] * h[n + 2] \)
    \end{enumerate} 

\end{homeworkProblem}


\begin{homeworkProblem}
    一个线性系统 \( S \) 的输入 \( x[n] \) 输出 \( y[n] \) 之间有如下关系:\\
    \[y[n] = \sum_{k = -\infty}^{\infty} x[k]g[n - 2k]\]
    其中 \( g[n] = u[n] - u[n - 4] \)。\\
    \begin{enumerate}[(1)]
        \item 当 \( x[n] = \delta[n - 1] \) 时,求 \( y[n] \)。
        \item 当 \( x[n] = \delta[n - 2] \) 时,求 \( y[n] \)。
        \item \( S \) 是线性时不变的吗?
        \item 当 \( x[n] = u[n] \) 时,求 \( y[n] \)。
    \end{enumerate}

\end{homeworkProblem}


\begin{homeworkProblem}
    考虑一个系统 \( S \),其输入 \( x[n] \) 与输出 \( y[n] \) 的关系为\\
    \[y[n] = x[n]\{g[n] + g[n - 1]\}\]
    \begin{enumerate}[(1)]
        \item 若对所有的 \( n \),\( g[n] = 1 \),证明 \( S \) 是时不变的。
        \item 若 \( g[n] = n \),证明 \( S \) 不是时不变的。
        \item 若 \( g[n] = 1 + (-1)^n \),证明 \( S \) 是时不变的。
    \end{enumerate}

\end{homeworkProblem}


\begin{homeworkProblem}
    设\\
    \[x[n] = \begin{cases}1, & 0 \leq n \leq 9 \\0, & \text{其他}\end{cases}\]
    且\\
    \[h[n] = \begin{cases}1, & 0 \leq n \leq N \\0, & \text{其他}\end{cases}\]
    其中 \( N \leq 9 \),是一个整数。已知 \( y[n] = x[n] * h[n] \) 且 \( y[4] = 5 \),\( y[14] = 0 \),试求 \( N \) 的值。

\end{homeworkProblem}


\begin{homeworkProblem}
    对下列各说法,判断是对还是错。\\
    \begin{enumerate}[(1)]
        \item 若 \( n < N_1 \) 时 \( x[n] = 0 \) 且 \( n < N_2 \) 时 \( h[n] = 0 \),那么 \( n < N_1 + N_2 \) 时 \( x[n] * h[n] = 0 \)。
        \item 若 \( y[n] = x[n] * h[n] \),则 \( y[n - 1] = x[n - 1] * h[n - 1] \)。
        \item 若 \( y(t) = x(t) * h(t) \),则 \( y(-t) = x(-t) * h(-t) \)。
        \item 若 \( t > T_1 \) 时 \( x(t) = 0 \) 且 \( t > T_2 \) 时 \( h(t) = 0 \),则 \( t > T_1 + T_2 \) 时 \( x(t) * h(t) = 0 \)。
    \end{enumerate}

\end{homeworkProblem}


\begin{homeworkProblem}
    计算并画出 \( y[n] = x[n] * h[n] \),其中\\
\[x[n] = \begin{cases}1, & 3 \leq n \leq 8 \\
0, & \text{其他}\end{cases}\]
\[h[n] = \begin{cases}1, & 4 \leq n \leq 15 \\
0, & \text{其他}\end{cases}\]
\end{homeworkProblem}


\begin{homeworkProblem}
    请写出两种噪声种类,他们有什么特点,以及对应的滤波方法

\end{homeworkProblem}


\begin{homeworkProblem}
    请写出两种损失函数,以及对应的公式和应用场景

\end{homeworkProblem}


\end{document}
