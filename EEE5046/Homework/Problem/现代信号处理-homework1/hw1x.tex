\documentclass[UTF8]{ctexart}

\usepackage{fancyhdr}
\usepackage{extramarks}
\usepackage{amsmath}
\usepackage{amsthm}
\usepackage{amsfonts}
\usepackage{tikz}
\usepackage[plain]{algorithm}
\usepackage{algpseudocode}
\usepackage{enumerate}

\usetikzlibrary{automata,positioning}

%
% Basic Document Settings
%

\topmargin=-0.45in
\evensidemargin=0in
\oddsidemargin=0in
\textwidth=6.5in
\textheight=9.0in
\headsep=0.25in

\linespread{1.1}

\pagestyle{fancy}
\lhead{\hmwkAuthorName\ \hmwkStudentID}
\chead{\hmwkClass: \hmwkTitle}
\rhead{\firstxmark}
\lfoot{\lastxmark}
\cfoot{\thepage}

\renewcommand\headrulewidth{0.4pt}
\renewcommand\footrulewidth{0.4pt}

\setlength\parindent{0pt}

%
% Create Problem Sections
%

\newcommand{\enterProblemHeader}[1]{
    \nobreak\extramarks{}{Problem \arabic{#1} continued on next page\ldots}\nobreak{}
    \nobreak\extramarks{Problem \arabic{#1} (continued)}{Problem \arabic{#1} continued on next page\ldots}\nobreak{}
}

\newcommand{\exitProblemHeader}[1]{
    \nobreak\extramarks{Problem \arabic{#1} (continued)}{Problem \arabic{#1} continued on next page\ldots}\nobreak{}
    \stepcounter{#1}
    \nobreak\extramarks{Problem \arabic{#1}}{}\nobreak{}
}

\setcounter{secnumdepth}{0}
\newcounter{partCounter}
\newcounter{homeworkProblemCounter}
\setcounter{homeworkProblemCounter}{1}
\nobreak\extramarks{Problem \arabic{homeworkProblemCounter}}{}\nobreak{}

\newenvironment{homeworkProblem}{
    \section{Problem \arabic{homeworkProblemCounter}}
    \setcounter{partCounter}{1}
    \enterProblemHeader{homeworkProblemCounter}
}{
    \exitProblemHeader{homeworkProblemCounter}
}

%
% Homework Details
%   - Title
%   - Due date
%   - Class
%   - Section/Time
%   - Instructor
%   - Author
%

\newcommand{\hmwkTitle}{Homework 1}
\newcommand{\hmwkDueDate}{Oct. 31, 2025}
\newcommand{\hmwkClass}{现代信号处理}
\newcommand{\hmwkStudentID}{这里写学号}
\newcommand{\hmwkAuthorName}{这里写姓名}

%
% Title Page
%

\title{
    \vspace{2in}
    \textmd{\textbf{\hmwkClass:\ \hmwkTitle}}\\
    \normalsize\vspace{0.1in}\small{Due\ on\ \hmwkDueDate\ }\\
    \vspace{3in}
}

\author{
	\textbf{\hmwkAuthorName}\\
	\textbf{学号\ \hmwkStudentID}
}
\date{}

\renewcommand{\part}[1]{\textbf{\large Part \Alph{partCounter}}\stepcounter{partCounter}}

%
% Various Helper Commands
%

% Useful for algorithms
\newcommand{\alg}[1]{\textsc{\bfseries \footnotesize #1}}

% For derivatives
\newcommand{\deriv}[1]{\frac{\mathrm{d}}{\mathrm{d}x} (#1)}

% For partial derivatives
\newcommand{\pderiv}[2]{\frac{\partial}{\partial #1} (#2)}

% Integral dx
\newcommand{\dx}{\mathrm{d}x}

% Alias for the Solution section header
\newcommand{\solution}{\textbf{\large Solution}}

% Probability commands: Expectation, Variance, Covariance, Bias
\newcommand{\E}{\mathrm{E}}
\newcommand{\Var}{\mathrm{Var}}
\newcommand{\Cov}{\mathrm{Cov}}
\newcommand{\Bias}{\mathrm{Bias}}

\begin{document}

\maketitle

\pagebreak

要求:latex\\
DDL:2025/10/31 下午24: 00分前提交pdf电子版\\
电子版以"homework1-姓名-学号"形式发送到12432643@mail.sustech.edu.cn邮箱\\


\begin{homeworkProblem}
    学习LaTeX的环境配置以及基本使用,在提供的作业模板题目下方给出答案,并在DDL前将最终生成的pdf文件发送到指定邮箱。(LaTeX编辑器选择TeXstudio, VScode, Overleaf 等都可以)

\end{homeworkProblem}


\begin{homeworkProblem}
    \begin{enumerate}[(1)]
        \item \(y(n) = x(-n)\)
        \item \(y(n) = x(n^2)\)
        \item \(y(n) = x^2(n)\)
        \item \(y(n) = x(n)sin(nw)\)
    \end{enumerate}
    试判断每一个系统是否具有线性、移不变性,并说明理由。\\
    \\

\end{homeworkProblem}


\begin{homeworkProblem}
    \begin{enumerate}[(1)]
        \item \(y(n) = \dfrac{{1}}{{N+1}}{\mathop{ \sum }\limits_{{k=0}}^{{N}}{x \left( n-k \right) }} \text {, N是大于零的整数}\)
        \item \(y(n) = x(-n)\)
        \item \(y(n) = x(n^2)\)
    \end{enumerate}
    试判定哪一个是因果系统,哪一个是非因果系统,并说明理由。\\
    \\

\end{homeworkProblem}


\begin{homeworkProblem}
    \begin{enumerate}[(1)]
        \item \(y(n) = \sum\limits_{{k=0}}^{{N-1}}a_{{k}}x(n-k) \),其中$a_0$,$a_1$,...,$a_{N-1}$为常数。\\
        \item \(y(n) = 2a cosw_0 y(n-1) - a^2 y(n-2) + x(n) - a cosw_0 x(n-1)\),其中$a$,$w_0$为常数。
    \end{enumerate}
    试求其单位抽样响应$h(n)$,并判断系统是否是稳定的。稳定的条件是什么?\\

\end{homeworkProblem}


\begin{homeworkProblem}
    证明系统的单位抽样响应在$n<0$时有$h(n) \equiv 0$, 且$x(n)$是因果信号,那么$y(n)$是因果信号。\\

\end{homeworkProblem}


\begin{homeworkProblem}
    $y(n)=ay(n-1)+x(n)$, $y(-1)=0$,试证明:
    \begin{enumerate}[(1)]
        \item 线性
        \item 移不变性
        \item 因果性
        \item $|a| \textless1$情况下的稳定性
    \end{enumerate}

\end{homeworkProblem}


\begin{homeworkProblem}
    设$x(nT_s)=e^{-nT_s}$为一指数函数,$n=0,1,2,...,\infty$,而$T_s$为抽样间隔,求$x(n)$的自相关函数$r_x(mT_s)$。\\

\end{homeworkProblem}


\begin{homeworkProblem}
    证明下列功率信号自相关函数的性质:
    \begin{enumerate}[(1)]
        \item 若$x(n)$是周期的,周期是$N$,则$r_x(m)=r_x(m+N)$
        \item 若$x(n)$是实的,则$r_x(m)=r_x(-m)$
        \item 若$x(n)$是复信号,则$r_x(m)=r_x^*(-m)$
    \end{enumerate}

\end{homeworkProblem}


\begin{homeworkProblem}
    简述:1.什么是过拟合?2.它会带来怎样的结果?3.数据增强为什么可以减轻过拟合现象?请简述一下常用手段。

\end{homeworkProblem}


\begin{homeworkProblem}
    简述:1. 卷积神经网络与全连接神经网络有什么区别?2. 什么是梯度消失,什么是梯度爆炸?3. ResNet是如何解决梯度消失问题的?

\end{homeworkProblem}


\end{document}
