\documentclass[UTF8]{ctexart}

\usepackage{fancyhdr}
\usepackage{extramarks}
\usepackage{amsmath}
\usepackage{amsthm}
\usepackage{amsfonts}
\usepackage{tikz}
\usepackage[plain]{algorithm}
\usepackage{algpseudocode}
\usepackage{enumerate}

\usetikzlibrary{automata,positioning}


%
% Basic Document Settings
%

\topmargin=-0.45in
\evensidemargin=0in
\oddsidemargin=0in
\textwidth=6.5in
\textheight=9.0in
\headsep=0.25in

\linespread{1.1}

\pagestyle{fancy}
\lhead{\hmwkAuthorName\ \hmwkStudentID}
\chead{\hmwkClass: \hmwkTitle}
\rhead{\firstxmark}
\lfoot{\lastxmark}
\cfoot{\thepage}

\renewcommand\headrulewidth{0.4pt}
\renewcommand\footrulewidth{0.4pt}

\setlength\parindent{0pt}

%
% Create Problem Sections
%

\newcommand{\enterProblemHeader}[1]{
    \nobreak\extramarks{}{Problem \arabic{#1} continued on next page\ldots}\nobreak{}
    \nobreak\extramarks{Problem \arabic{#1} (continued)}{Problem \arabic{#1} continued on next page\ldots}\nobreak{}
}

\newcommand{\exitProblemHeader}[1]{
    \nobreak\extramarks{Problem \arabic{#1} (continued)}{Problem \arabic{#1} continued on next page\ldots}\nobreak{}
    \stepcounter{#1}
    \nobreak\extramarks{Problem \arabic{#1}}{}\nobreak{}
}

\setcounter{secnumdepth}{0}
\newcounter{partCounter}
\newcounter{homeworkProblemCounter}
\setcounter{homeworkProblemCounter}{1}
\nobreak\extramarks{Problem \arabic{homeworkProblemCounter}}{}\nobreak{}

\newenvironment{homeworkProblem}{
    \section{Problem \arabic{homeworkProblemCounter}}
    \setcounter{partCounter}{1}
    \enterProblemHeader{homeworkProblemCounter}
}{
    \exitProblemHeader{homeworkProblemCounter}
}

%
% Homework Details
%   - Title
%   - Due date
%   - Class
%   - Section/Time
%   - Instructor
%   - Author
%

\newcommand{\hmwkTitle}{Homework 3}
\newcommand{\hmwkDueDate}{Nov.30, 2025}
\newcommand{\hmwkClass}{现代信号处理}
\newcommand{\hmwkStudentID}{这里写学号}
\newcommand{\hmwkAuthorName}{某某某}

%
% Title Page
%

\title{
    \vspace{2in}
    \textmd{\textbf{\hmwkClass:\ \hmwkTitle}}\\
    \normalsize\vspace{0.1in}\small{Due\ on\ \hmwkDueDate\ }\\
    \vspace{3in}
}

\author{
	\textbf{\hmwkAuthorName}\\
	\textbf{学号\ \hmwkStudentID}
}
\date{}

\renewcommand{\part}[1]{\textbf{\large Part \Alph{partCounter}}\stepcounter{partCounter}\\}

%
% Various Helper Commands
%

% Useful for algorithms
\newcommand{\alg}[1]{\textsc{\bfseries \footnotesize #1}}

% For derivatives
\newcommand{\deriv}[1]{\frac{\mathrm{d}}{\mathrm{d}x} (#1)}

% For partial derivatives
\newcommand{\pderiv}[2]{\frac{\partial}{\partial #1} (#2)}

% Integral dx
\newcommand{\dx}{\mathrm{d}x}

% Alias for the Solution section header
\newcommand{\solution}{\textbf{\large Solution}}

% Probability commands: Expectation, Variance, Covariance, Bias
\newcommand{\E}{\mathrm{E}}
\newcommand{\Var}{\mathrm{Var}}
\newcommand{\Cov}{\mathrm{Cov}}
\newcommand{\Bias}{\mathrm{Bias}}

\begin{document}

\maketitle

\pagebreak

要求:latex\\
DDL:2025/11/30 下午23: 59分前提交pdf电子版\\
电子版以"homework3-姓名-学号"形式发送到12332151@mail.sustech.edu.cn邮箱\\
\begin{homeworkProblem}
	An unknown parameter $\theta$ influences the outcome of an experiment which is modeled by the random variable $x$. The PDF of $x$ is
	\begin{align*} p(x ; \theta)=\frac{1}{\sqrt{2 \pi}} \exp \left[-\frac{1}{2}(x-\theta)^{2}\right]
	\end{align*}
	A series of experiments is performed, and $x$ is found to always be in the interval [97, 103]. As a result, the investigator concludes that $\theta$ must have been 100. Is this assertion correct?\\
	\newline
	
\end{homeworkProblem}

\begin{homeworkProblem}
	It is desired to estimate the value of a DC level $A$ in $\mathrm{WGN}$ or
	$$
	x[n]=A+w[n] \quad n=0,1, \ldots, N-1
	$$
	where $w[n]$ is zero mean and uncorrelated, and each sample has variance $\sigma^2=1$. Consider the two estimators
	$$
	\begin{aligned}
	\hat{A} &=\frac{1}{N} \sum_{n=0}^{N-1} x[n] \\
	\check{A} &=\frac{1}{N+2}\left(2 x[0]+\sum_{n=1}^{N-2} x[n]+2 x[N-1]\right) .
	\end{aligned}
	$$
	Which one is better? Does it depend on the value of $A$ ?\\
	\newline
	
\end{homeworkProblem}

\begin{homeworkProblem}
	The data $\{ x[0], x[1], ... ,x[N-1] \}$ are observed where the $x[n]'s$ are independent and identically distributed (IID) as $N(0,\sigma ^2)$. We wish to estimate the variance $\sigma ^2$ as
	\begin{align*}
	\hat {\sigma ^2} = \dfrac{1}{N} \sum_{n=0}^{N-1} x^2[n]
	\end{align*}
	Is this an unbiased estimator? Find the variance of $\hat{\sigma ^2}$ and examine what happens as $N \rightarrow \infty$.\\
	\newline
	
\end{homeworkProblem}

\begin{homeworkProblem}
	Two samples $\{ x[0], x[1] \}$ are independently observed from a $N(0, \sigma^2)$ distribution. The estimator
	\begin{align*}
	\hat{\sigma ^2} = \dfrac{1}{2} (x^2[0] + x^2[1])
	\end{align*}
	is unbiased. Find the PDF of $\hat{\sigma^2}$ to determine if it is symmetric about $\sigma^2$.\\
	\newline
	
\end{homeworkProblem}

\begin{homeworkProblem}
    Independent bivariate Gaussian samples $\{x[0], x[1], ..., x[N-1]\}$ are observed. Each observation is a $2 \times 1$ vector which is distributed as $x[n] \sim \mathcal N(0,C)$ and
    \begin{align*}
    	C=\begin{bmatrix}1&\rho\\\rho&1\end{bmatrix} .
    \end{align*}
    Find the CRLB for the correlation coefficient $\rho$.\\
\end{homeworkProblem}

\begin{homeworkProblem}
    If $x[n]=r^n+w[n]$ for $n=0, 1, ..., N-1$ are observed, where $w[n]$ is WGN with variance $\sigma^2$ and $r$ is to be estimated, find the CRLB. Does an efficient estimator exist and if so find its variance?\\
\end{homeworkProblem}

\begin{homeworkProblem}
    Using the results of Example 3.13, determine the best range estimation accuracy of sonar if
    \begin{align*}
        s(t) = \left\{ {\begin{array}{*{20}{c}}
        {1 - 100\left| {t - 0.01} \right|}\\
        0
        \end{array}} \right.\begin{array}{*{20}{c}}
        {}\\
        {}
        \end{array}\begin{array}{*{20}{c}}
        {0 \le t \le 0.02}\\
        {otherwise.}
        \end{array}
    \end{align*}
    Let $N_0/2=10^{-6}$ and $c=1500 m/s$.
    \\    
\end{homeworkProblem}
\newpage
\begin{homeworkProblem}
	
	\begin{figure}[ht]
		\centering
		\includegraphics[width=0.8\linewidth]{hw14.jpg}
		%\caption{}\label{1}
	\end{figure}
\end{homeworkProblem}

\end{document}
