\documentclass[UTF8]{ctexart}

\usepackage{fancyhdr}
\usepackage{extramarks}
\usepackage{amsmath}
\usepackage{amsthm}
\usepackage{amsfonts}
\usepackage{tikz}
\usepackage{pgfplots} 
\pgfplotsset{compat=1.17} % 适配版本,避免兼容性警告
\usepackage[plain]{algorithm}
\usepackage{algpseudocode}
\usepackage{enumerate}

\usetikzlibrary{automata,positioning}

% 基本文档设置
\topmargin=-0.45in
\evensidemargin=0in
\oddsidemargin=0in
\textwidth=6.5in
\textheight=9.0in
\headsep=0.25in

\linespread{1.1}

\pagestyle{fancy}
\lhead{\hmwkAuthorName\ \hmwkStudentID}
\chead{\hmwkClass: \hmwkTitle}
\rhead{\firstxmark}
\lfoot{\lastxmark}
\cfoot{\thepage}

\renewcommand\headrulewidth{0.4pt}
\renewcommand\footrulewidth{0.4pt}

\setlength\parindent{0pt}

% 问题部分设置
\newcommand{\enterProblemHeader}[1]{
    \nobreak\extramarks{}{Problem \arabic{#1} continued on next page\ldots}\nobreak{}
    \nobreak\extramarks{Problem \arabic{#1} (continued)}{Problem \arabic{#1} continued on next page\ldots}\nobreak{}
}

\newcommand{\exitProblemHeader}[1]{
    \nobreak\extramarks{Problem \arabic{#1} (continued)}{Problem \arabic{#1} continued on next page\ldots}\nobreak{}
    \stepcounter{#1}
    \nobreak\extramarks{Problem \arabic{#1}}{}\nobreak{}
}

\setcounter{secnumdepth}{0}
\newcounter{partCounter}
\newcounter{homeworkProblemCounter}
\setcounter{homeworkProblemCounter}{1}
\nobreak\extramarks{Problem \arabic{homeworkProblemCounter}}{}\nobreak{}

\newenvironment{homeworkProblem}{
    \section{Problem \arabic{homeworkProblemCounter}}
    \setcounter{partCounter}{1}
    \enterProblemHeader{homeworkProblemCounter}
}{
    \exitProblemHeader{homeworkProblemCounter}
}

% 作业信息
\newcommand{\hmwkTitle}{Homework 5}
\newcommand{\hmwkDueDate}{Jan. 30, 2026}
\newcommand{\hmwkClass}{现代信号处理}
\newcommand{\hmwkStudentID}{12345678}
\newcommand{\hmwkAuthorName}{助教}

% 标题页
\title{
    \vspace{2in}
    \textmd{\textbf{\hmwkClass:\ \hmwkTitle}}\\
    \normalsize\vspace{0.1in}\small{Due\ on\ \hmwkDueDate\ }\\
    \vspace{3in}
}

\author{
	\textbf{\hmwkAuthorName}\\
	\textbf{学号\ \hmwkStudentID}
}
\date{}

\renewcommand{\part}[1]{\textbf{\large Part \Alph{partCounter}}\stepcounter{partCounter}}

% 辅助命令
\newcommand{\alg}[1]{\textsc{\bfseries \footnotesize #1}}
\newcommand{\deriv}[1]{\frac{\mathrm{d}}{\mathrm{d}x} (#1)}
\newcommand{\pderiv}[2]{\frac{\partial}{\partial #1} (#2)}
\newcommand{\dx}{\mathrm{d}x}
\newcommand{\solution}{\textbf{\large Solution}}
\newcommand{\E}{\mathrm{E}}
\newcommand{\Var}{\mathrm{Var}}
\newcommand{\Cov}{\mathrm{Cov}}
\newcommand{\Bias}{\mathrm{Bias}}

\begin{document}

\maketitle

\pagebreak

要求:latex\\
DDL:2026/1/4 下午23:59分前提交pdf电子版\\
电子版以"homework5-姓名-学号"形式发送到12332186@mail.sustech.edu.cn邮箱\\


\begin{homeworkProblem}
    Given observations $x[n]$ for $n = 0, 1, \dots, N-1$, where the samples are i.i.d. and distributed according to $U[\theta_1, \theta_2]$, find a sufficient statistic for $\boldsymbol{\theta} = [\theta_1, \theta_2]^T$.
\end{homeworkProblem}
    
\begin{homeworkProblem}
    For $n = 0, 1, \dots, N-1$, suppose $x[n] = A r^n + w[n]$, where $A$ is an unknown parameter, $r$ is an unknown constant, and $w[n]$ is white noise with zero mean and variance $\sigma^2$. Find the BLUE of $A$ and its minimum variance. Does the minimum variance tend to zero as $N \to \infty$?
\end{homeworkProblem}
    
\begin{homeworkProblem}
    The observed i.i.d. samples $\{x[0], x[1], \dots, x[N-1]\}$ follow the distributions below:
    \begin{enumerate}
        \item[a.] Laplace:
        \[
            p(x[n]; \mu) = \frac{1}{2} \exp\left[-|x[n] - \mu|\right]
        \]
        \item[b.] Gaussian:
        \[
            p(x[n]; \mu) = \frac{1}{\sqrt{2\pi}} \exp\left[-\frac{1}{2}(x[n] - \mu)^2\right]
        \]
    \end{enumerate}
    Find the BLUE of the mean $\mu$ for both cases. Also, explain the MVU estimator of $\mu$.
\end{homeworkProblem}
    
\begin{homeworkProblem}
    Assume that the observed signal is $x[n] = A s[n] + w[n]$, for $n = 0, 1, \dots, N-1$, where $w[n]$ is noise with zero mean and covariance matrix $\mathbf{C}$, and $s[n]$ is a known signal. The amplitude $A$ is the parameter to be estimated. Find the BLUE of $A$. Discuss what happens if the characteristic vector of $\mathbf{C}$ is $\mathbf{s} = [s[0]\; s[1]\; \dots\; s[N-1]]^T$. Also, find the minimum variance.
\end{homeworkProblem}
    
\begin{homeworkProblem}
    Prove the linearity property of the BLUE with respect to linear transformations of $\boldsymbol{\theta}$. Specifically, if we wish to estimate
    \[
        \boldsymbol{\alpha} = \mathbf{B} \boldsymbol{\theta} + \mathbf{b},
    \]
    where $\mathbf{B}$ is a known $p \times p$ invertible matrix and $\mathbf{b}$ is a known $p \times 1$ vector, prove that the BLUE of $\boldsymbol{\alpha}$ is given by
    \[
        \hat{\boldsymbol{\alpha}} = \mathbf{B} \hat{\boldsymbol{\theta}} + \mathbf{b},
    \]
    where $\hat{\boldsymbol{\theta}}$ is the BLUE of $\boldsymbol{\theta}$. Assume the data model $\mathbf{x} = \mathbf{H} \boldsymbol{\theta} + \mathbf{w}$, where $E(\mathbf{w}) = \mathbf{0}$ and $E(\mathbf{w} \mathbf{w}^T) = \mathbf{C}$. Hint: Substitute $\boldsymbol{\theta}$ for $\boldsymbol{\alpha}$ in the data model.
\end{homeworkProblem}
    
\begin{homeworkProblem}
    For the general linear model
    \[
        \mathbf{x} = \mathbf{H} \boldsymbol{\theta} + \mathbf{s} + \mathbf{w},
    \]
    where $\mathbf{s}$ is a known $N \times 1$ vector, $E(\mathbf{w}) = \mathbf{0}$, and $E(\mathbf{w} \mathbf{w}^T) = \mathbf{C}$, find the BLUE of $\boldsymbol{\theta}$.
\end{homeworkProblem}
    
\begin{homeworkProblem}
    We observe $N$ i.i.d. samples from the following PDFs:
    
    \begin{enumerate}
        \item[a.] Gaussian:
        \[
            p(x; \mu) = \frac{1}{\sqrt{2\pi}} \exp\left[-\frac{1}{2}(x - \mu)^2\right]
        \]
    
        \item[b.] Exponential:
        \[
            p(x; \lambda) = 
            \begin{cases} 
                \lambda \exp(-\lambda x) & x > 0 \\
                0 & x < 0 
            \end{cases}
        \]
    \end{enumerate}
    
    In each case, find the MLE of the unknown parameter and verify that it indeed maximizes the likelihood function. Is the estimator meaningful?
\end{homeworkProblem}
    
\begin{homeworkProblem}
    The following is the formal definition of a consistent estimator: If for any given $\epsilon > 0$, it satisfies
    \[
        \lim_{N \to \infty} \Pr\left\{ |\hat{\theta} - \theta| > \epsilon \right\} = 0,
    \]
    then the estimator $\hat{\theta}$ is consistent.
    
    Prove that for the problem of estimating a DC level $A$ in white Gaussian noise with known variance, the sample mean is a consistent estimator. Hint: Use Chebyshev's inequality.
\end{homeworkProblem}

\end{document}