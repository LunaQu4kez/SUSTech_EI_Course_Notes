\documentclass[UTF8]{ctexart}

\usepackage{fancyhdr}
\usepackage{extramarks}
\usepackage{amsmath}
\usepackage{amsthm}
\usepackage{amsfonts}
\usepackage{tikz}
\usepackage[plain]{algorithm}
\usepackage{algpseudocode}
\usepackage{enumerate}

\usetikzlibrary{automata,positioning}


%
% Basic Document Settings
%

\topmargin=-0.45in
\evensidemargin=0in
\oddsidemargin=0in
\textwidth=6.5in
\textheight=9.0in
\headsep=0.25in

\linespread{1.1}

\pagestyle{fancy}
\lhead{\hmwkAuthorName\ \hmwkStudentID}
\chead{\hmwkClass: \hmwkTitle}
\rhead{\firstxmark}
\lfoot{\lastxmark}
\cfoot{\thepage}

\renewcommand\headrulewidth{0.4pt}
\renewcommand\footrulewidth{0.4pt}

\setlength\parindent{0pt}

%
% Create Problem Sections
%

\newcommand{\enterProblemHeader}[1]{
    \nobreak\extramarks{}{Problem \arabic{#1} continued on next page\ldots}\nobreak{}
    \nobreak\extramarks{Problem \arabic{#1} (continued)}{Problem \arabic{#1} continued on next page\ldots}\nobreak{}
}

\newcommand{\exitProblemHeader}[1]{
    \nobreak\extramarks{Problem \arabic{#1} (continued)}{Problem \arabic{#1} continued on next page\ldots}\nobreak{}
    \stepcounter{#1}
    \nobreak\extramarks{Problem \arabic{#1}}{}\nobreak{}
}

\setcounter{secnumdepth}{0}
\newcounter{partCounter}
\newcounter{homeworkProblemCounter}
\setcounter{homeworkProblemCounter}{1}
\nobreak\extramarks{Problem \arabic{homeworkProblemCounter}}{}\nobreak{}

\newenvironment{homeworkProblem}{
    \section{Problem \arabic{homeworkProblemCounter}}
    \setcounter{partCounter}{1}
    \enterProblemHeader{homeworkProblemCounter}
}{
    \exitProblemHeader{homeworkProblemCounter}
}

%
% Homework Details
%   - Title
%   - Due date
%   - Class
%   - Section/Time
%   - Instructor
%   - Author
%

\newcommand{\hmwkTitle}{Homework 4}
\newcommand{\hmwkDueDate}{Dec.14, 2025}
\newcommand{\hmwkClass}{现代信号处理}
\newcommand{\hmwkStudentID}{这里写学号}
\newcommand{\hmwkAuthorName}{某某某}

%
% Title Page
%

\title{
    \vspace{2in}
    \textmd{\textbf{\hmwkClass:\ \hmwkTitle}}\\
    \normalsize\vspace{0.1in}\small{Due\ on\ \hmwkDueDate\ }\\
    \vspace{3in}
}

\author{
	\textbf{\hmwkAuthorName}\\
	\textbf{学号\ \hmwkStudentID}
}
\date{}

\renewcommand{\part}[1]{\textbf{\large Part \Alph{partCounter}}\stepcounter{partCounter}\\}

%
% Various Helper Commands
%

% Useful for algorithms
\newcommand{\alg}[1]{\textsc{\bfseries \footnotesize #1}}

% For derivatives
\newcommand{\deriv}[1]{\frac{\mathrm{d}}{\mathrm{d}x} (#1)}

% For partial derivatives
\newcommand{\pderiv}[2]{\frac{\partial}{\partial #1} (#2)}

% Integral dx
\newcommand{\dx}{\mathrm{d}x}

% Alias for the Solution section header
\newcommand{\solution}{\textbf{\large Solution}}

% Probability commands: Expectation, Variance, Covariance, Bias
\newcommand{\E}{\mathrm{E}}
\newcommand{\Var}{\mathrm{Var}}
\newcommand{\Cov}{\mathrm{Cov}}
\newcommand{\Bias}{\mathrm{Bias}}

\begin{document}

\maketitle

\pagebreak

要求:latex\\
DDL:2025/12/14 下午23: 59分前提交pdf电子版\\
电子版以"homework4-姓名-学号"形式发送到12432643@mail.sustech.edu.cn邮箱\\
\begin{homeworkProblem}
    Consider the continuous-time signal $x(t)$ defined as
    \begin{align*} x(t) = \frac{\sin 2\pi t}{\pi t}
    \end{align*}
    Prove that $x(t)$ is square integrable, but is not absolutely integrable.\\
    \newline
    
\end{homeworkProblem}

\begin{homeworkProblem}
    We wish to estimate the amplitudes of exponential signals in noise. The observed data is given by
    \begin{align*} x[n] = \sum_{i=1}^{p} A_i r_i^n + w[n], \quad n=0, 1, \dots, N-1
    \end{align*}
    where $w[n]$ is white Gaussian noise (WGN) with variance $\sigma^2$. Find the MVU estimator of the amplitudes and their covariance. Evaluate your results for the specific case where $p=2$, $r_1=1$, $r_2=-1$, and $N$ is even.\\
    \newline
    
\end{homeworkProblem}

\begin{homeworkProblem}
    Consider the observation matrix
    \begin{align*} \mathbf{H} = \begin{bmatrix} 1 & 1 \\ 1 & 1 \\ 1 & 1 + \epsilon \end{bmatrix}
    \end{align*}
    where $\epsilon$ is small. Compute $(\mathbf{H}^T \mathbf{H})^{-1}$, and examine what happens as $\epsilon \to 0$. If the observation vector is $\mathbf{x} = [2 \ \ 2 \ \ 2]^T$, find the MVU estimator. Describe what happens as $\epsilon \to 0$.\\
    \newline
    
\end{homeworkProblem}
\newpage
\begin{homeworkProblem}
    In practice, we sometimes encounter the linear model $\mathbf{x} = \mathbf{H}\boldsymbol{\theta} + \mathbf{w}$, where $\mathbf{H}$ is composed of random variables. Suppose we ignore this difference and use the usual estimator
    \begin{align*} \hat{\boldsymbol{\theta}} = (\mathbf{H}^T \mathbf{H})^{-1} \mathbf{H}^T \mathbf{x}
    \end{align*}
    where we assume the specific realization of $\mathbf{H}$ is known. Prove that if $\mathbf{H}$ and $\mathbf{w}$ are independent, then the mean and covariance of $\hat{\boldsymbol{\theta}}$ are
    \begin{align*} E(\hat{\boldsymbol{\theta}}) &= \boldsymbol{\theta} \\
    \mathbf{C}_{\hat{\theta}} &= \sigma^2 E_H [(\mathbf{H}^T \mathbf{H})^{-1}]
    \end{align*}
    \newline
    
\end{homeworkProblem}

\begin{homeworkProblem}
    Suppose we observe a fading signal in noise. We view this fading signal as being derived from another "on" or "off" signal. For example, consider a DC level in WGN, i.e., $x[n] = A + w[n]$, for $n=0, 1, \dots, N-1$. When the signal fades, the data model becomes
    \begin{align*}
        x[n] = \begin{cases}
            A + w[n] & n = 0, 1, \dots, M-1 \\
            w[n] & n = M, M+1, \dots, N-1
        \end{cases}
    \end{align*}
    where the probability of fading is $\epsilon$. Assume we know when the signal has undergone fading. Use the results of Problem 4 to determine the estimator of $A$ and its variance, and compare this result with the case of no fading.\\
    \newline
    
\end{homeworkProblem}

\begin{homeworkProblem}
    The IID observations $x[n]$ for $n=0, 1, \dots, N-1$ have the exponential PDF
    \begin{align*}
        p(x[n]; \lambda) = \begin{cases}
            \lambda \exp(-\lambda x[n]) & x[n] > 0 \\
            0 & x[n] < 0
        \end{cases}
    \end{align*}
    Find a sufficient statistic for $\lambda$.\\
    \newline
    
\end{homeworkProblem}
\newpage
\begin{homeworkProblem}
    Assume $x[n]$ is the result of a Bernoulli trial (coin toss), with
    \begin{align*}
        \Pr\{x[n]=1\} &= \theta \\
        \Pr\{x[n]=0\} &= 1 - \theta
    \end{align*}
    and $N$ IID observations are made. Assuming the Neyman-Fisher Factorization Theorem holds for discrete random variables, find a sufficient statistic for $\theta$. Then, assuming completeness, find the MVU estimator of $\theta$.\\
    \newline
    
\end{homeworkProblem}

\begin{homeworkProblem}
    Consider a sinusoidal signal of known frequency in WGN, i.e.,
    \begin{align*} x[n] = A \cos 2\pi f_0 n + w[n] \quad n=0, 1, \dots, N-1
    \end{align*}
    where $w[n]$ is WGN with variance $\sigma^2$. Find the MVU estimators for the following parameters:
    \begin{enumerate}
        \item Amplitude $A$, assuming $\sigma^2$ is known;
        \item Amplitude $A$ and noise variance $\sigma^2$.
    \end{enumerate}
    You may assume the sufficient statistic is complete.\\
    \newline

\end{homeworkProblem}

\end{document}
